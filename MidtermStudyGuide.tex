\documentclass[]{article}

\usepackage{hyperref}
\usepackage{amsmath}
\usepackage{listings}

\lstset{
  basicstyle=\itshape,
  xleftmargin=3em,
  literate={->}{$\rightarrow$}{2}
}

\begin{document}

\title{CPSC 327 Artificial Intelligence Spring 2015: Mid-Term Study Guide}

\maketitle

\section{Name 5 Attributes of Intelligence}
Use the reading, class notes, your paper, and my comments to prepare 5 attributes of intelligence that you can write for this part of the test.
\vspace{1in}
\section{Grammars}
Given the following grammar:
\begin{lstlisting}
A -> B '&' C
B -> C '%' C
C -> '[a-z]'
\end{lstlisting}

which of the following are valid?
\begin{enumerate}
\item a \% b \& c
\item b \% c \% d
\item x \& y \% z
\item x \% y \& z
\end{enumerate}

\newpage

Given the following grammar:
\begin{lstlisting}
U -> V ' ' U | V
V ->  '[a-z]+' | '(' U ')'
\end{lstlisting}

which of the following are valid?
\begin{enumerate}
\item I shot an elephant in my pajamas
\item all roads lead to rome
\item amen
\item what is the flying speed of an unburdened sparrow (european or african)
\end{enumerate}

\section{Proofs in The Wumpus World}
Keep these logic rules in your back pocket:

\begin{itemize}
\item Or Commutativity: $A \vee B \equiv B \vee A$.
\item And Commutativity: $A \wedge B \equiv B \wedge A$.
\item Or Associativity: $A \vee (B \vee C) \equiv (A \vee B) \vee C$.
\item And Associativity: $A \wedge (B \wedge C) \equiv (A \wedge B) \wedge C$.
\item Double Negation: $\neg \neg A \equiv A$.
\item Contraposition: $A \rightarrow B \equiv \neg B \rightarrow \neg A$.
\item Implication Elimination: $A \rightarrow B \equiv \neg A \vee B$.
\item Biconditional Elimination: $A \leftrightarrow B \equiv (A \rightarrow B) \wedge (B \rightarrow A)$.
\item DeMorgan's Law 1: $\neg (A \wedge B) \equiv \neg A \vee \neg B$.
\item DeMorgan's Law 2: $\neg (A \vee B) \equiv \neg A \wedge \neg B$.
\item Distribute Or: $A \vee (B \wedge C) \equiv (A \vee B) \wedge (A \vee C)$.
\item Distribute And: $A \wedge (B \vee C) \equiv (A \wedge B) \vee (A \wedge C)$.
\end{itemize}
and these inference rules:
\begin{itemize}
\item Modus Ponens: \[\frac{A \rightarrow B, A}{B}\].
\item And Elimination: \[\frac{A \wedge B}{A}\]
\end{itemize}

\newpage

1.  Given the rules:
\begin{description}
\item[R1:] $\neg P_{1,1}$.
\item[R2:] $B_{1,1} \leftrightarrow (P_{1,2} \vee P_{2,1})$.
\item[R3:] $B_{2,1} \leftrightarrow (P_{1,1} \vee P_{2,2} \vee P_{3,1})$. 
\end{description}

we then visit [1,1] and [2,1] and learn the following facts:
\begin{description}
\item[R4:] $\neg B_{1,1}$.
\item[R5:] $B_{2,1}$.
\end{description}

A.  Prove that there is no pit at location [1,2].

\vspace{1in}
Now move to location [1,2] and learn the following fact:
\begin{description}
\item[R5:] $\neg B_{1,2}$.
\end{description}

B.  Prove that there is a pit in location [3,1]

\newpage
\section{prolog Programs}
\subsection{What, if anything, is wrong with the following prolog clause?}
sibling(A,B):- parent(X,A), parent(X,B).
\vspace{5mm}

  What will result if it is run against family.pl?
\vspace{1in}



\subsection{Write a set of prolog clauses to define a fibonacci sequence}
\vspace{1in}




\end{document}